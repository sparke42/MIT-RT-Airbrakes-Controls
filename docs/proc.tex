\documentclass{amsdtx}
\usepackage[margin=.75in]{geometry}
\usepackage{graphicx}
\usepackage{float}
\usepackage{eqnarray,amsmath}
\usepackage{amssymb}
\usepackage{url}
\usepackage[dvipsnames]{xcolor}
\usepackage{cancel}
\usepackage{multicol}
\usepackage[Glenn]{fncychap}
%\usepackage{hyperref}
\definecolor{b}{HTML}{3C78D8}
\definecolor{g}{HTML}{38761D}
\definecolor{r}{HTML}{AB0000}

\usepackage{listings}

\definecolor{codegreen}{HTML}{38761D}
\definecolor{codeblue}{HTML}{000076}
\definecolor{codegray}{rgb}{0.2,0.2,0.2}
\definecolor{codepurple}{HTML}{AB0000}
\definecolor{backcolour}{rgb}{0.95,0.95,0.95}

\lstdefinestyle{mystyle}{
    backgroundcolor=\color{white},   
    commentstyle=\color{codegreen},
    keywordstyle=\color{codepurple},
    numberstyle=\ttfamily\scriptsize\color{codeblue},
    stringstyle=\color{b},
    basicstyle=\ttfamily\footnotesize,
    breakatwhitespace=false,         
    breaklines=true,                 
    captionpos=b,                    
    keepspaces=true,                 
    numbers=left,                    
    numbersep=5pt,                  
    showspaces=false,                
    showstringspaces=false,
    showtabs=false,                  
    tabsize=4
}
\lstset{style=mystyle}

\newcommand{\bs}[1]{\boldsymbol{#1}}
\usepackage{bm}

\newcolumntype{M}[1]{>{\centering\arraybackslash}m{#1}}
\newcolumntype{N}{@{}m{0pt}@{}}

\newcommand{\vfs}[1]{\ensuremath \textbf{\textit{f}}\kern-0.1cm_{s_{#1}}}
\newcommand{\fs}[1]{\ensuremath f\kern-0.07cm_{s_{#1}}}
\newcommand{\F}{\ensuremath \textbf{F}\kern-0.07cm}
\newcommand{\cc}{\ensuremath\mu_{cc}}
\newcommand{\cg}{\ensuremath\mu_{cg}}
\usepackage{enumitem}

\usepackage[T2A,OT1]{fontenc}
\usepackage[utf8]{inputenc}
\usepackage[russian,english]{babel}

\title{\sc Airbrakes Control Algorithm}
\author{\sc Marc D Nichitiu, Akash Krishna | 8 January 2026}
\date{\sc MIT Rocket Team Simulations}
\begin{document}
\maketitle    
\section{Abstract}
We wish to choose a target apogee altitude using a variable-area airbrake system. Airbrakes are deployed using a servomotor that can expose a particular fraction of the Airbrake area to the freestream, thereby allowing the drag to depend on time. 

This algorithm seeks to enable a rocket having some default without-Airbrakes apogee altitude to lower this altitude to a close but particular height, to be chosen by the designer. Overall flow is as follows:
\begin{enumerate}
	\item A linear approximation (coefficient $\alpha$) of the vertical acceleration of the rocket as a function of time is characterized.
	\item A non-Airbrakes apogee altitude is determined according to this approximation.
	\item The difference between default apogee altitude and desired apogee altitude is calculated according to the pre-determined altitude choice. Let us denote its magnitude $\Delta x$.
	\item The Airbrakes' deployed area $A(t)$ is calculated from $\Delta x$, according to the ramp shape in \S2 Eq. (\ref{13}). 
	\item Once Airbrakes are activated, they are to be extended according to $A(t)$, with a PIDF controller on position accuracy and acceleration. 
\end{enumerate}

\section{Derivation: Energy Conservation}
Consider the rocket to undergo purely vertical motion. In the case in which airbrakes are never deployed, the rocket will reach an apogee height of $x_{\rm apog}$, corresponding to a gravitational potential energy of $mgx_{\rm apog}$. Airbrakes deployment, with area some $A(t)$, will do work on the rocket:
\begin{align}
	W = \int F~dx = \int \frac{C_D\rho A(t)\dot x^2}{2}~dx 
\end{align}
This work will contribute to some loss of altitude in the apogee, given by
\begin{align}
	\int \frac{C_D\rho A(t)\dot x^2}{2}~dx = mg(\Delta x)
\end{align}
We recall that $\dot x = dx/dt$, and so $dx = \dot x dt$:
\begin{align}
		\int \frac{C_D\rho A(t)\dot x^3}{2}~dt = mg(\Delta x)
\end{align}
The governing differential equation is
\begin{align}
	\ddot x = -g-B\dot x^2
\end{align}
where
\begin{align}
	B = \frac{C_{D_{\rm rocket}}\rho A_{\rm ref}}{2m}
\end{align}
We approximate that acceleration varies linearly in time, of the form
\begin{align}
	\ddot x = -g + \alpha(t-t_{\rm apog})\label{6}
\end{align}
We solve for $\dot x$:
\begin{align}
	\dot x = \sqrt{-\frac{g +\ddot x}{B}}
\end{align}
Thus, 
\begin{align}
	\dot x^3 = \left(-\frac{g +\ddot x}{B}\right)^{3/2}
\end{align}
Substituting in (\ref{6}) gives
\begin{align}
	\dot x^3 = \left(-\alpha\frac{t_{\rm apog}-t}{B}\right)^{3/2}
\end{align}
Thus, the integral becomes
\begin{align}
	\frac{C_D\rho\alpha^{3/2}}{2B^{3/2}}\int A(t)(t_{\rm apog}-t)^{3/2} ~dt =  mg(\Delta x)
\end{align}
We move all terms to the other side:
\begin{align}
	\int A(t)(t_{\rm apog}-t)^{3/2} ~dt = \frac{2mg(\Delta x)B^{3/2}}{C_D\rho\alpha^{3/2}}
\end{align}
Let $\eta$ be the right side expression, such that
\begin{align}
		\int A(t)(t_{\rm apog}-t)^{3/2} ~dt = \eta 
\end{align}
Values of $\eta$ for different altitudes are given below according to current Zephyrus statistics:
\begin{table}[H]
	\centering
	\renewcommand{\arraystretch}{1.3}
	\begin{tabular}{|c|c|}
	\hline $\bf \Delta x$ & $\bs \eta$ \\\hline\hline
	300~m & 1.875 \\
	200~m & 1.249 \\
	100~m & 0.625\\\hline
	\end{tabular}
\end{table}
\noindent We are free to choose the bounds of $A(t)$'s nonzero interval, as well as the kind of function $A(t)$ is. We will choose a ramp-like function, of the form
\begin{align}
	A(t) = \left\{\begin{array}{ll}
2A_0(t-t_0) &\displaystyle t \in \left[t_0	,t_0+\frac 12\right] \\[1em]
A_0 & \displaystyle t \geq t_0+\frac 12
\end{array}
\right.\label{13}
\end{align}
Thus, 
\begin{align}
	\frac{\eta}{A_0} &= \int\limits_{t_0}^{t_0+1/2}2(t-t_0)(t_{\rm apog}-t)^{3/2}~dt + \int\limits_{t_0+1/2}^{t_{\rm apog}}(t_{\rm apog}-t)^{3/2}~dt
\end{align}
We perform a change of variables from $\tau = t_{\rm apog}-t$ which gives $d\tau = -dt$ and $t = t_{\rm apog} - \tau$. The integration limits become
\begin{align}
	t_0 \to t_{\rm apog} -t_0\quad;\quad t_0+\frac 12 \to t_{\rm apog} - t_0 - \frac 12\quad;\quad t_{\rm apog} \to 0
\end{align}
Thus,
\begin{align}
	\frac{\eta}{A_0} &= -2\int \limits_{t_{\rm apog} -t_0}^{t_{\rm apog} - t_0 -1/2}(t_{\rm apog} - \tau -t_0)\tau^{3/2}~d\tau - \int\limits_{t_{\rm apog} - t_0 -1/2}^{0}\tau^{3/2}~d\tau  \\
	&= 2\int \limits_{t_{\rm apog} - t_0 -1/2}^{t_{\rm apog} -t_0}(t_{\rm apog} - \tau -t_0)\tau^{3/2}~d\tau  + \int\limits_{0}^{t_{\rm apog} - t_0 -1/2}\tau^{3/2}~d\tau
\end{align}
We expand the expression into three integrals, and solve them one by one. The first is
\begin{align}
	2(t_{\rm apog} -t_0)\int \limits_{t_{\rm apog} - t_0 -1/2}^{t_{\rm apog} -t_0}\tau^{3/2}~d\tau &= 2(t_{\rm apog} -t_0)\left(\frac{\tau^{5/2}}{5/2}\right)\biggr|_{t_{\rm apog} - t_0 -1/2}^{t_{\rm apog} -t_0} \\
	&= \frac 45(t_{\rm apog} -t_0)\left((t_{\rm apog} -t_0)^{5/2} - (t_{\rm apog} - t_0 -1/2)^{5/2}\right)
\end{align}
The second integral evaluates to
\begin{align}
	-2\int \limits_{t_{\rm apog} - t_0 -1/2}^{t_{\rm apog} -t_0}\tau^{5/2}~d\tau &= -2\left(\frac{\tau^{7/2}}{7/2}\right)\biggr|_{t_{\rm apog} - t_0 -1/2}^{t_{\rm apog} -t_0} \\
	&= -\frac 47\left((t_{\rm apog} -t_0)^{7/2} - (t_{\rm apog} - t_0 -1/2)^{7/2}\right)
\end{align}
The third integral evaluates to
\begin{align}
	\int\limits_{0}^{t_{\rm apog} - t_0 -1/2}\tau^{3/2}~d\tau &= \left(\frac{\tau^{5/2}}{5/2}\right)\biggr|_0^{t_{\rm apog}-t_0-1/2}  = \frac 25(t_{\rm apog}-t_0-1/2)^{5/2}
\end{align}
Thus, $A_0$ evaluates to
\begin{align}
	\kern-.3cm\boxed{\frac{\eta}{\frac 45(t_{\rm apog} -t_0)\left((t_{\rm apog} -t_0)^{\frac 52} - (t_{\rm apog} - t_0 -\frac 12)^{\frac 52}\right) - \frac 47\left((t_{\rm apog} -t_0)^{\frac 72} - (t_{\rm apog} - t_0 -\frac 12)^{\frac 72}\right) + \frac 25(t_{\rm apog}-t_0-\frac 12)^{\frac 52}}}
\end{align}
The maximum area of $A(t)$ is $0.0066\rm~m^2$.  A table of results is below. Note that apogee time is 31.89 seconds.
\begin{table}[H]
\centering
\renewcommand{\arraystretch}{1.4}
\begin{tabular}{|c||c|c|c|}
\hline $t_0$ & $\Delta x = 300\rm~m$& $\Delta x = 200\rm~m$& $\Delta x = 100\rm~m$\\\hline\hline
15 s & 63\% & 42\% & 21\% \\\hline
17.5 s & 94\% & 63\% & 32\% \\\hline
19 s & \color r124\% & 83\% & 42\% \\\hline
20 s & \color r153\% & \color r102\% & 51\% \\\hline
\end{tabular}


~\\\caption{Values of $\frac{A_0}{A_{\rm max}}$ requisite for the labeled altitude difference, according to Zephyrus Test Launch Simuilations.}
	
\end{table}



\section{Implementation, Characterizations}


\end{document}













